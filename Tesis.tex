% Para el glosario
\immediate\write18{makeindex \jobname.nlo -s nomencl.ist -o \jobname.nls}

% Cambiar la siguiente línea a oneside para versión online
\documentclass[12pt,twoside]{book}

% Fuentes y codificación
\usepackage[T1]{fontenc}
\usepackage[utf8]{inputenc}
\usepackage{lmodern}
\usepackage{microtype}
\usepackage{mathrsfs} %* Proporciona un comando \mathscr, en lugar de
                      %* sobrescribir el comando estándar \mathcal, como en
                      %* calrsfs.

% Formato de papel y marcas de cortado
\usepackage[letter,center,pdflatex]{crop} %* Agregar la opción cam para marcas
                                          %* de cortado (crop marks)
\usepackage{geometry}
  \geometry{
    paperwidth=17.409cm,
    paperheight=23.665cm,
    textheight=19.274cm,
    textwidth=13.998cm,
    right=1.2cm, % comentar para versión online
    includehead
  }

% Idioma
\usepackage[spanish,mexico]{babel}

% AMS-LaTeX
\usepackage{amsmath,amssymb} % Símbolos matemáticos
\usepackage{amsthm} % Entornos de teoremas
% \usepackage{amsfonts}

% Espaciado entre líneas (útil para matrices, necesario para \setstretch)
\usepackage{setspace}

% Índices
\usepackage{imakeidx}

% Inclusión de imágenes
\usepackage{graphicx}

% Encabezados y pies de Página
\usepackage{fancyhdr}

% Gestión de colores
\usepackage{xcolor}

% Creación de figuras
\usepackage{tikz}
  \usetikzlibrary{calc}
% Para dibujar gráficas
\tikzstyle{edge}=[shorten <=1pt, shorten >=1pt, >=stealth, line width=0.8pt]
\tikzstyle{curveEdge}=[shorten <=1pt, shorten >=1pt, >=stealth, bend right=20,
                       bend left=20, line width=1pt]
\tikzstyle{vertex}=[circle, draw, minimum size=5pt, inner sep=0pt,
                    outer sep=0pt]
\tikzstyle{blackV}=[circle, fill=black, draw, minimum size=5pt, inner sep=0pt,
                    outer sep=0pt]

% Modificadores para objetos flotantes
\usepackage{float}

% Hipervínculos en referencias cruzadas
\usepackage[hidelinks]{hyperref}
	\providecommand\phantomsection{}

% Referencias cruzadas con nombres automáticos
\usepackage[capitalize]{cleveref}

% Columnas múltiples
\usepackage{multirow}

% Formato para tablas y matrices
\usepackage{array}

% Relleno automático de texto
\usepackage{lipsum}

% Glosario de símbolos (Agregar sólo si son muchos)
\usepackage[intoc,refpage]{nomencl}
	\makenomenclature
	\renewcommand{\nomname}{Glosario de símbolos}
	\makeatletter
		\newcommand \Dotfill {\leavevmode \cleaders \hb@xt@ .77em{\hss .\hss }\hfill \kern \z@}
	\makeatother
	\def\pagedeclaration#1{\Dotfill\hyperlink{page.#1}{#1}}
	\newcommand{\nom}[2]{\nomenclature{#1}{#2}}
	\setlength{\nomlabelwidth}{2.5cm}
	\setlength{\nomitemsep}{0.15cm}
\usepackage{etoolbox}
	\renewcommand\nomgroup[1]{\item[\ifstrequal{#1}{$ $}{ }{}]}

% Definición de Teoremas y relacionados
\newtheorem{teorema}{Teorema}[chapter]
\newtheorem{lema}[teorema]{Lema}
\newtheorem{proposicion}[teorema]{Proposición}
\newtheorem{corolario}[teorema]{Corolario}
\newtheorem{conjetura}[teorema]{Conjetura}
\newtheorem{problema}[teorema]{Problema}

\theoremstyle{definition}
\newtheorem{definicion}[teorema]{Definición}
\newtheorem{observacion}[teorema]{Observación}
\newtheorem{ejemplo}[teorema]{Ejemplo}

% Macros para estilo de definición e indexado simultaneo
\newcommand{\Def}[1]{\index{#1}\emph{\textbf{#1}}}
  % Por ejemplo \Def{graph}
\newcommand{\DeepDef}[2]{\index{#1}{\bf\em{#2}}}
  % Por ejemplo \DeepDef{graph!bipartite graph}{bipartite

% Comandos auxiliares
\newcommand{\Pclass}{\mbox{P}}
\newcommand{\NPclass}{\mbox{NP}}

\interfootnotelinepenalty=100 % Un valor >>100 evita partir pies de página


%%%%%%%%%%%%%%%%%%%%%%%%%%%%%%%%%%%%%%%%%%%%%%%%%%%%%%%%%%%%%%%%%%%%%%%%%%%%%%%%
%%%%%%%%%%%%%%%%%%%%%%%%%%%%%%%%%%%%%%%%%%%%%%%%%%%%%%%%%%%%%%%%%%%%%%%%%%%%%%%%
%%%%%%%%%%%%%%%%%%%%%%%%%%%%%%%%%%%%%%%%%%%%%%%%%%%%%%%%%%%%%%%%%%%%%%%%%%%%%%%%
%%%%%%%%%%%%%%%%%%%%%%%%%%%%%%%%%%%%%%%%%%%%%%%%%%%%%%%%%%%%%%%%%%%%%%%%%%%%%%%%
%%%%%%%%%%%%%%%%%%%%%%%%%%%%%%%%%%%%%%%%%%%%%%%%%%%%%%%%%%%%%%%%%%%%%%%%%%%%%%%%

\makeindex[intoc]

\begin{document}

  \begin{hyphenrules}{spanish}

    \frontmatter  % % % % % % % % % % % % % % % % % % % % % % % % % % % % % % %

      \pagestyle{fancy}
      \renewcommand{\chaptermark}[1]{\markboth{#1}{}}
      \renewcommand{\sectionmark}[1]{\markright{\thesection\ #1}}
      \fancyhf{}
      \fancyhead[LE,RO]{\bfseries\thepage}
      \fancyhead[LO]{\bfseries\rightmark}
      \fancyhead[RE]{\bfseries\leftmark}
      \renewcommand{\headrulewidth}{0.5pt}
      \renewcommand{\footrulewidth}{0pt}
      \addtolength{\headheight}{0.5pt}
      \fancypagestyle{plain}{
      \fancyhead{}
      \renewcommand{\headrulewidth}{0pt}}

      % % % % % % % % % % % % % % % % % % % % % % % % % % % % % % % % % % % % % % % %
%
% Esqueleto para la portada de las tesis de licenciatura
% para la Facultad de Ciencias de la UNAM
%
% % % % % % % % % % % % % % % % % % % % % % % % % % % % % % % % % % % % % % % %

\newcommand{\titulo}[1]{\def\eltitulo{#1}}
\newcommand{\carrera}[1]{\def\lacarrera{#1}}
\newcommand{\nombre}[1]{\def\elnombre{#1}}    %* Del alumno
\newcommand{\director}[1]{\def\eldirector{#1}}  %* De tesis
\newcommand{\fecha}[1]{\def\lafecha{#1}}

% Llena los siguientes datos (usando mayúsculas y minúsculas según corresponda)
% Estos datos aparecerán en la portada y los encabezados

\titulo{\emph{El título de la tesis.}}
\nombre{\uppercase{NOMBRE1~NOMBRE2~APELLIDO1~APELLIDO2}}
\carrera{MATEMÁTICO} %* Corresponde al tí­tulo otorgado -Matemático-
										 %* NO AL NOMBRE DE LA CARRERA -Matemáticas-
\director{\uppercase{LIC.~NOMBRE1~NOMBRE2~APELLIDO1~APELLIDO2}}
\fecha{AAAA}

% % % % % % % % % % % % % % % % % % % % % % % % % % % % % % % % % % % % % % % %

\thispagestyle{empty}

%% Barra izquierda - Escudos
\hskip-1.5cm
\begin{minipage}[c][10cm][s]{3cm}
  \begin{center}
    \includegraphics[height=2.6cm]{escudo-unam}\\[10pt]
    \hskip2pt\vrule width2pt height13cm\hskip1mm
    \vrule width1pt height13cm\\[10pt]
    \includegraphics[height=2.6cm]{escudo-ciencias}
  \end{center}
\end{minipage}\quad
%
%% Barra derecha - Tí­tulos
\begin{minipage}[c][9.5cm][s]{10cm}
  \begin{center}
    % Barra superior
    {\large \scshape Universidad Nacional Autónoma de México}
    \vspace{.3cm}
    \hrule height2pt
    \vspace{.1cm}
    \hrule height1pt
    \vspace{.3cm}
    {\scshape Facultad de Ciencias}

    % Titulo del trabajo
    \vspace{3cm}

    {\Large \eltitulo}

    \vspace{3cm}

    % Tipo de trabajo
    \makebox[8cm][s]{\Huge T E S I S}\\[8pt]
    QUE PARA OBTENER EL TÍTULO DE:\\[3pt]
    \mbox{}\lacarrera\\[13pt]
    PRESENTA:\\[3pt]
    \elnombre

    \vspace{2cm}

    {\small DIRECTOR DE TESIS:\\ \eldirector}

    \vspace{2cm}

    \lafecha

  \end{center}
\end{minipage}
 % Portada y datos generales
      \begin{quote}
\begin{tabular}{lll}
%
1.Datos del alumno           &                                         \\
Apellido paterno             & APELLIDO1                               \\
Apellido materno             & APELLIDO2                               \\
Nombre(s)                    & NOMBRE1 NOMBRE2                         \\
Teléfono                     & XX XXXX XXXXX                           \\
Universidad                  & Universidad Nacional Autónoma de México \\
Facultad o escuela           & Facultad de Ciencias                    \\
Carrera                      & Matemáticas                             \\
Número de cuenta             & XXXXXXXXX                               \\
                             &                                         \\
%
2. Datos del tutor           &                                         \\
Grado                        & Xxx.                                    \\
Nombre(s)                    & NOMBRE1 NOMBRE2                         \\
Apellido paterno             & APELLIDO1                               \\
Apellido materno             & APELLIDO2                               \\
                             &                                         \\
%
3. Datos del sinodal 1       &                                         \\
Grado                        & Xxx.                                    \\
Nombre(s)                    & NOMBRE1 NOMBRE2                         \\
Apellido paterno             & APELLIDO1                               \\
Apellido materno             & APELLIDO2                               \\
                             &                                         \\
%
4. Datos del sinodal 2       &                                         \\
Grado                        & Xxx.                                    \\
Nombre(s)                    & NOMBRE1 NOMBRE2                         \\
Apellido paterno             & APELLIDO1                               \\
Apellido materno             & APELLIDO2                               \\
                             &                                         \\
%
%
5. Datos del sinodal 3       &                                         \\
Grado                        & Xxx.                                    \\
Nombre(s)                    & NOMBRE1 NOMBRE2                         \\
Apellido paterno             & APELLIDO1                               \\
Apellido materno             & APELLIDO2                               \\
                             &                                         \\
%
6. Datos del sinodal 4       &                                         \\
Grado                        & Xxx.                                    \\
Nombre(s)                    & NOMBRE1 NOMBRE2                         \\
Apellido paterno             & APELLIDO1                               \\
Apellido materno             & APELLIDO2                               \\
                             &                                         \\
%
7. Datos del trabajo escrito &                                         \\
Título                       & El título de la tesis.                  \\
Número de páginas            & XX                                      \\
Año                          & AAAA                                    \\
%
\end{tabular}
\end{quote}
 % Datos del alumno, tutor, sinodales y del trabajo
      %Formato de agradcecimiento breve
\thispagestyle{empty}

\ \\

\newpage

\thispagestyle{empty}

\begin{center}
\begin{minipage}[l][10cm][t]{14cm}


	\vspace{5.5cm}

	\begin{flushright}

	{\em Para XXXXX,\quad}

	{\em por tal o cual motivo.\quad}

	\end{flushright}

\end{minipage}
\end{center}

\newpage

\thispagestyle{empty}

\ \\
 % Agradecimientos

      \tableofcontents{\setcounter{tocdepth}{3}}

      \include{Intro} % Introducción al trabajo

    \mainmatter % % % % % % % % % % % % % % % % % % % % % % % % % % % % % % % %

      \chapter{Introducción a la Teoría de Gráficas}

\lipsum
 % Introducción al área de estudio
      \chapter{El título del capítulo 2}

\lipsum
 % Tema 1
      \chapter{El título del capítulo 3}

\lipsum
 % Resultados
      \chapter*{Conclusiones}%                                   |
\phantomsection
\addcontentsline{toc}{chapter}{Conclusiones}

\lipsum
 % Conclusiones del trabajo

    \backmatter % % % % % % % % % % % % % % % % % % % % % % % % % % % % % % % %

      \phantomsection
\addcontentsline{toc}{chapter}{Biblografía}

\begin{thebibliography}{999}

\bibitem{GT-Bondy}
	Bondy, J. A., \& Murty, U. S. (2008). {\em Graph theory}. Graduate texts in mathematics, 1(244), ALL-ALL.

\end{thebibliography}

      	%\cleardoublepage
      	\markboth{\MakeUppercase\nomname}{\MakeUppercase\nomname}
      \printnomenclature
      \printindex

  \end{hyphenrules}

\end{document}
